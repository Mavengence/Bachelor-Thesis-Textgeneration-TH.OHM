\chapter{Prototyp}\label{ch:method}

In this chapter, we're actually using some code!

\begin{lstlisting}[language=Python,caption={This is an example of inline listing},captionpos=b]
x = 1
if x == 1:
    # indented four spaces
    print("x is 1.")

\end{lstlisting}

You can also include listings from a file directly:

\lstinputlisting[language=Python,caption={This is an example of included listing},captionpos=b]{listings/example.py}

\section{Zielsetzung}

Image Captioning

\section{Fachkonzept}

Fachkonzept

\subsection{Struktur}

The different steps of Text Generation

\begin{itemize}
\item Importing Dependencies
\item Loading the Data
\item Creating Character/Word mappings
\item Data Preprocessing
\item Modelling
\item Generating text
\end{itemize}

\subsection{Neuronales Netz}

LSTM
https://www.analyticsvidhya.com/blog/2017/12/fundamentals-of-deep-learning-introduction-to-lstm/


Experimenting with different models

\begin{itemize}
\item A more trained model
\item A deeper model
\item A wider model
\item A gigantic model
\end{itemize}

\subsection{Prozessmodellierung}

Funktionen etc.

\subsection{Datenflussmodellierung}

Diagramm

\section{Implementierung}

Code

\section{Evaluation}

Print Ergebnisse

Bild

Image Caption