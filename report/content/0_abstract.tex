\thispagestyle{empty}
\section*{Preface I}
\label{sec:prolog_1}

The following thesis was created during my 7th and last semester at the University of Applied Science - Georg Simon OHM. 

Within my last three semesters I realized, that my major interest among all IT related topics is artificial intelligence.

I worked out together with my professor \textit{Prof. Dr. Alfred Holl} a method-matrix for the entire structure of this paper. Without his cooperative support overseas, which I was studying abroad at the City University of Hong Kong, this thesis would have not been possible for me.

Even though Natural Language Processing is just a subfield of machine learning itself, the current state-of-the-art research is far beyond what I can research within a bachelor thesis. In this way, I decided to write my thesis about the subfield \textit{textgeneration} within NLP. My state-of-the-art research includes all \textit{hot topics} within NLP and my prototyp focuses only on the textgeneration part, to dive deeper into what NLP is able to accomplish in the year 2020.

\newpage

\section*{Preface Il}
\label{sec:prolog_2}

My interest started basically with my IT project, in which my team and I programmed an autonomously driving remote control car with a deep neural network and a Raspberry Pi 3. From this first project on, I selected all my elective courses to be related with machine learning or data science in any possible way. I wanted to further increase my knowledge, so I searched for a website which provides courses related to AI. I found \textit{www.udacity.com}, which offers courses  in cooperation with top IT companys, such as Google, Airbnb or Microsoft. Out of curiosity I bought the course \textit{Natural Language Processing}. After successfully finishing it, I was encouraged to write my bachelor thesis in a subfield of \textit{Natural Language Processing}. 

For my research I encountered a lot of recently published and old papers from \textit{https://arxiv.org/}. To read through the papers requires a lot of prior knowledge, which I learned in my abroad semester in Hong Kong. 

Machine Learning and more specifically NLP is not an intuitive study. I provided the common terminologys from top researchers and tried to make the entry into this field as smooth as possible, if the reader has no prior knowledge about this topic.

I still recommend some basic linear algebra and calculus knowledge to understand the formulas more easily.

Thank you very much for reading.

\newpage

\section*{Abstract}
\label{sec:abstract}

-- At the end , finally finished :) --
