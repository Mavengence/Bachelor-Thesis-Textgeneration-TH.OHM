\chapter{State of the Art}\label{ch:sota}

\section{Background and Theory}
\subsection{Prerequisites}
\subsection{Encoder}
\subsection{Decoder}


\section{History of Text Generation}\label{ss:history}

nlg-survey-long

\subsection{Text Generation Tasks}
raditionally, the nlg problem of converting input data into output text was addressed by splitting it up into a number of subproblems. The following six are frequently found in many nlg systems (Reiter and Dale, 1997, 2000); their role is illustrated in Figure 1:
\begin{itemize}
	\item Content determination: Deciding which information to include in the text under construction
	\item Text structuring: Determining in which order information will be pre- sented in the text
	\item Sentence aggregation: Deciding which information to present in individual sentences
	\item Lexicalisation: Finding the right words and phrases to express informa- tion
	\item Referring expression generation: Selecting the words and phrases to iden- tify domain objects
	\item Linguistic realisation: Combining all words and phrases into well-formed sentences
\end{itemize}

\subsection{Architectures and Approaches}

nlg-survey-long Kapitel 3

\begin{itemize}
	\item Rule-based, modular approaches
	\item Planning-based approaches
	\item Data-driven approaches
\end{itemize}


\subsection{Neural Text Generation}
NTG with Supervised Learning
NTG with Reinforcement Learning
NTG with GAN's

\section{Current Trends in Text Summarization Technology}\label{ss:trends}

\subsection{Summarization Factors}
Single Doc - Multi Doc
Input Factors
Purpose Factors
output factors
neural-text-summary

\subsection{Extractive}
\subsection{Abstractive}
\subsection{Combinational Approach}
\subsection{Reinforcement Learning}
\subsection{Evaluation}
ROGUE